\documentclass[journal,12pt,twocolumn]{IEEEtran}

\usepackage{setspace}
\usepackage{gensymb}

\singlespacing


\usepackage[cmex10]{amsmath}

\usepackage{amsthm}

\usepackage{mathrsfs}
\usepackage{txfonts}
\usepackage{stfloats}
\usepackage{bm}
\usepackage{cite}
\usepackage{cases}
\usepackage{subfig}

\usepackage{longtable}
\usepackage{multirow}

\usepackage{enumitem}
\usepackage{mathtools}
\usepackage{steinmetz}
\usepackage{tikz}
\usepackage{circuitikz}
\usepackage{verbatim}
\usepackage{tfrupee}
\usepackage[breaklinks=true]{hyperref}
\usepackage{graphicx}
\usepackage{tkz-euclide}

\usetikzlibrary{calc,math}
\usepackage{listings}
    \usepackage{color}                                            %%
    \usepackage{array}                                            %%
    \usepackage{longtable}                                        %%
    \usepackage{calc}                                             %%
    \usepackage{multirow}                                         %%
    \usepackage{hhline}                                           %%
    \usepackage{ifthen}                                           %%
    \usepackage{lscape}     
\usepackage{multicol}
\usepackage{chngcntr}

\DeclareMathOperator*{\Res}{Res}

\renewcommand\thesection{\arabic{section}}
\renewcommand\thesubsection{\thesection.\arabic{subsection}}
\renewcommand\thesubsubsection{\thesubsection.\arabic{subsubsection}}

\renewcommand\thesectiondis{\arabic{section}}
\renewcommand\thesubsectiondis{\thesectiondis.\arabic{subsection}}
\renewcommand\thesubsubsectiondis{\thesubsectiondis.\arabic{subsubsection}}


\hyphenation{op-tical net-works semi-conduc-tor}
\def\inputGnumericTable{}                                 %%

\lstset{
%language=C,
frame=single, 
breaklines=true,
columns=fullflexible
}
\begin{document}


\newtheorem{theorem}{Theorem}[section]
\newtheorem{problem}{Problem}
\newtheorem{proposition}{Proposition}[section]
\newtheorem{lemma}{Lemma}[section]
\newtheorem{corollary}[theorem]{Corollary}
\newtheorem{example}{Example}[section]
\newtheorem{definition}[problem]{Definition}

\newcommand{\BEQA}{\begin{eqnarray}}
\newcommand{\EEQA}{\end{eqnarray}}
\newcommand{\define}{\stackrel{\triangle}{=}}
\bibliographystyle{IEEEtran}
\providecommand{\mbf}{\mathbf}
\providecommand{\pr}[1]{\ensuremath{\Pr\left(#1\right)}}
\providecommand{\qfunc}[1]{\ensuremath{Q\left(#1\right)}}
\providecommand{\sbrak}[1]{\ensuremath{{}\left[#1\right]}}
\providecommand{\lsbrak}[1]{\ensuremath{{}\left[#1\right.}}
\providecommand{\rsbrak}[1]{\ensuremath{{}\left.#1\right]}}
\providecommand{\brak}[1]{\ensuremath{\left(#1\right)}}
\providecommand{\lbrak}[1]{\ensuremath{\left(#1\right.}}
\providecommand{\rbrak}[1]{\ensuremath{\left.#1\right)}}
\providecommand{\cbrak}[1]{\ensuremath{\left\{#1\right\}}}
\providecommand{\lcbrak}[1]{\ensuremath{\left\{#1\right.}}
\providecommand{\rcbrak}[1]{\ensuremath{\left.#1\right\}}}
\theoremstyle{remark}
\newtheorem{rem}{Remark}
\newcommand{\sgn}{\mathop{\mathrm{sgn}}}
\providecommand{\abs}[1]{\left\vert#1\right\vert}
\providecommand{\res}[1]{\Res\displaylimits_{#1}} 
\providecommand{\norm}[1]{\left\lVert#1\right\rVert}
%\providecommand{\norm}[1]{\lVert#1\rVert}
\providecommand{\mtx}[1]{\mathbf{#1}}
\providecommand{\mean}[1]{E\left[ #1 \right]}
\providecommand{\fourier}{\overset{\mathcal{F}}{ \rightleftharpoons}}
%\providecommand{\hilbert}{\overset{\mathcal{H}}{ \rightleftharpoons}}
\providecommand{\system}{\overset{\mathcal{H}}{ \longleftrightarrow}}
    %\newcommand{\solution}[2]{\textbf{Solution:}{#1}}
\newcommand{\solution}{\noindent \textbf{Solution: }}
\newcommand{\cosec}{\,\text{cosec}\,}
\providecommand{\dec}[2]{\ensuremath{\overset{#1}{\underset{#2}{\gtrless}}}}
\newcommand{\myvec}[1]{\ensuremath{\begin{pmatrix}#1\end{pmatrix}}}
\newcommand{\mydet}[1]{\ensuremath{\begin{vmatrix}#1\end{vmatrix}}}
\numberwithin{equation}{subsection}
\makeatletter
\@addtoreset{figure}{problem}
\makeatother
\let\StandardTheFigure\thefigure
\let\vec\mathbf
\renewcommand{\thefigure}{\theproblem}
\def\putbox#1#2#3{\makebox[0in][l]{\makebox[#1][l]{}\raisebox{\baselineskip}[0in][0in]{\raisebox{#2}[0in][0in]{#3}}}}
     \def\rightbox#1{\makebox[0in][r]{#1}}
     \def\centbox#1{\makebox[0in]{#1}}
     \def\topbox#1{\raisebox{-\baselineskip}[0in][0in]{#1}}
     \def\midbox#1{\raisebox{-0.5\baselineskip}[0in][0in]{#1}}
\vspace{3cm}
\title{Assignment 1}
\author{MUKUL KUMAR YADAV\\ EE20RESCH14003}
\maketitle
\newpage
\bigskip
\renewcommand{\thefigure}{\theenumi}
\renewcommand{\thetable}{\theenumi}
 Download Latex codes from here
\begin{lstlisting}
https://github.com/EE20RESCH14003/Assignment-1_4
\end{lstlisting}
%

%

%
\section{\textbf{Question No. 62}}
A line perpendicular to the line segment joining the points (1,0) and (2,3) divides it into the ratio 1:n. Find the equation of the line.

\subsection{\textbf{Solution}}

\begin{center}
    
\begin{tikzpicture}

\coordinate [label=right:$A$] (A) at (1,0);
\coordinate [label=right:$B$] (B) at (2,3);
\coordinate [label=right:$P$] (P) at (1.5,1.5);
\coordinate [label=right:$R$] (R) at (-2,4);
\coordinate [label=left:$O$] (O) at (0,0);
\draw (A) -- (B);
\draw (P) -- (R);
\draw [dash dot] (O) - -(A);
\draw [dash dot] (O) - -(P);
\draw [dash dot] (O) - -(B);
\end{tikzpicture} 
\end{center} 
Given that
\begin{align}
A=\myvec{1 \\0}  and \ B= \myvec{2 \\3}
\end{align}
Let 
\begin{align}
P= \myvec{x \\y}  and \ origin \ O = \myvec{0 \\0}
\end{align}
Since the line RP intersect the line AB in 1:n ration, then 
\begin{align}
 \frac{AP}{PB} = \frac{1}{n} \\
 \implies PB = nAP  
\end{align}
Vector equation of line AB is 
\begin{align}
\vec{r} = \vec{A}+\lambda\vec{d} 
\end{align}

A is point where the line passes, d is direction vector, and $\lambda$ is constant. 

Direction vector 
\begin{align}
\vec{d} = \myvec{2\\3}-\myvec{1\\0} = \myvec{1\\3}
\end{align}
Therefore, vector equation of line AB is 
\begin{align}
\vec{r}=\myvec{1\\0}+\lambda\myvec{1\\3}
\end{align}
 $\vec{r}$ is the point on line, and the point moves on the line as $\lambda$ varies. Here, the line PR divides the line AB in 1:n ratio. 
\begin{align}
\myvec{x\\y} = \myvec{1\\0}+\frac{1}{n+1}\myvec{1\\3} = \myvec{\frac{n+2}{n+1}\\\frac{3}{n+1}}
\end{align}
Therefore, coordinate of point $P=\myvec{\frac{n+2}{n+1}\\\frac{3}{n+1}}$
\\
Equation of line PR is 
\begin{align}
\vec{r}=\vec{P}+\lambda\vec{d}\\
\implies \vec{r}=\myvec{\frac{n+2}{n+1}\frac{3}{n+1}}+\lambda\myvec{d1\\d2}
\end{align}

Since both the lines AB and PR are perpendicular to each other, then dot product of direction vectors will be zero.
\begin{align}
\myvec{1\\3}^T\myvec{d1\\d2}=\myvec{0\\0} \\
\implies  \myvec{d1\\d2} = \myvec{3\\-1}
\end{align}

Finally, vector equation of line PR will be
\begin{align}
\vec{r}=\myvec{\frac{n+2}{n+1}\\\frac{3}{n+1}}+\lambda\myvec{3\\-1}
\end{align}
\end{document}